\documentclass[11pt]{article}
\usepackage[top = 1in, bottom = 1in, left =1in, right = 1in]{geometry}
\usepackage{graphicx}
\usepackage{amsmath}
\usepackage{amssymb}
\usepackage{enumerate}
\usepackage{hyperref}
\usepackage{fancyhdr}\pagestyle{fancy}

\lhead{Lewis \& Clark}
\chead{Math 131, Fall 2019}
\rhead{}

\begin{document}
\begin{center}
\Large {Calculus I, MATH 131}
\end{center}

\begin{tabular}[h!]{l l}
\noindent
\textbf{Instructor:} & Lucas Van Meter\\
\textbf{Email:} & lvanmeter@lclark.edu\\
\textbf{Office Hours:} & MF 9:30-11:00am, TTh 2:00-3:00pm, BoDine 304 \\
\textbf{Website: } & 
 \url{https://lucasvanmeter.github.io/classes/MATH131/index.html}
\end{tabular}

\section*{Overview}

The objective of this course is to equip the student with an understanding of and ability to apply differential calculus, and to introduce the student to integral calculus. The main topics of the course are limits, derivatives \& rates of change, applications of derivatives, methods for computing derivatives, cumulative effect \& definite integrals, and the Fundamental Theorem of Calculus. For each topic we will examine the conceptual understanding, formal mathematical definition, and graphical interpretation.

\section*{Course Learning Objectives}

Some of the goals for this course are to be able to:
\begin{enumerate}\setlength\itemsep{0em}
  \item Understand and make appropriate use of the mathematical vocabulary and notation used in this course,
  \item Communicate your problem solving strategies clearly and accurately in both written and oral form,
  \item Analyze and interpret graphical representations,
  \item Analyze, synthesize, and evaluate information,
  \item Apply mathematical techniques to formulate and solve problems,
  \item Interpret and explain solutions to quantitative problems.
  \item Learn in a group and independent context, and
  \item Approach mathematics with confidence.
\end{enumerate}

\section*{Class Structure} This course is structured to give you many points of contact with the material and encourage effective study habits. Every week will look roughly like this:
\begin{enumerate}
    \item Before each new section students read the section and answer the pre-class questions.
    \item Class time will mostly be spent with active learning activities like working in groups, discussing as a class, or working individually. I will not be lecturing on all the material in the book so it is essential to have read the section to be prepared for class.
    \item Homework will be due a couple of days after we discuss a topic in class, usually most Mondays and Thursdays.
    \item Each week we will have a quiz (or midterm) covering the previous weeks (or multiple weeks) material.
    \item Students can also rework assignments to make up some missed points up to a week after they are due. 
\end{enumerate}
These five steps are designed to give students continual feedback, many opportunities and ways to engage with the material, and opportunity to learn from their mistakes and show improvement.

\section*{Strategies for Success}

Learning mathematics is not a spectator sport. It requires active engagement and practice. Here are some behaviors of succesfull students:
\begin{itemize}
    \item Come to class prepared to ask questions, actively participate in discussions, and bravely embrace the mistakes that all of us make when we make an honest effort. 
    \item Outside of class read the textbook, complete the homework, and spend additional time doing practice problems to study for quizzes and exams.
    \item Actively seek out help when you feel confused or unsure about how to learn effectively. Here are some people that can help you:
    \begin{itemize}
        \item Me! Come to my office hours, I am delighted to talk with you one on one outside of class.
        \item Your classmates! Form a study group, ask each other questions, gain deeper understanding by explaining things to each other.
        \item The Symbolic and Quantitative Resource Center (SQRC), located in  Howard Hall 134, offers free peer tutoring on a drop-in basis. It's also a great place to meet with your study group. See \url{http://college.lclark.edu/departments/mathematical_sciences/sqrc/}.
        \item SAAB tutoring is another free tutoring service. See \url{https://college.lclark.edu/academics/support/advising/saab-tutoring/}
    \end{itemize}
\end{itemize}

\section*{Office Hours}

I highly encourage you to come to my office hours with or without specific questions. It is the best time for me to give you one-on-one attention for anything ranging from homework questions to ways to succeed in the course.

I have scheduled office hours Monday, Friday from 9:30-11:00am and Tuesday, Thursday from 2:00-3:00pm. You can also contact me via email or during class to schedule an appointment outside of those times.

\section*{Grading and Assesment}

This class uses a ``mastery based assesment'' model designed to encourage continual improvement and eventual mastery of the course goals. You will be allowed to resubmit any work you turn in (except the final for time reasons) up to a week after receiving it back to receive up to half the missed points back. See each assignment type for details

I will use the standard 90/80/70/60 scale, as shown below. However, your final grade may take other factors into consideration, such as effort, improvement, and active participation.

\begin{center}
  \begin{tabular}{ccccc}
     A: 93-100\% & A-: 90-92.99\% & B+: 88-89.99\% & B: 83-87.99\% & B-: 80-82.99\% 
     \\
     C+: 78-79.99\% & C: 73-77.99\% & C-: 70-72.99\% & D: 60-69.99\%	&  F: 0-59.99\%
  \end{tabular}
\end{center}

Your course grade will be determined based on your percentage of points received with these approximate weights:

\begin{center}
	\begin{tabular}{cc}
		\underline{Category} & \underline{Percentage}
	    \\
		Pre-class questions	 & 5\%
		\\
		Homework(23)	 & 25\%
		\\
		Quizzes(8)	 & 20\%
		\\
		Midterm	I & 15 \%
		\\
		Midterm	II & 15 \%
		\\
		Final   	& 20 \%
		\\
		\hline
		Total: & 100\%
	\end{tabular}
\end{center}

\subsection*{Pre-Class Questions}

Pre-class questions serve two purposes: (i) they help you engage with and reflect on the class material before class and (ii) they help me evaluate where the class is and what specific topics we might need to focus on during class.

Pre-class questions will be posted and turned in on moodle. Pre-class questions are graded by complete if they demonstrate an honest engagement with the material and should not take more than 10 minutes after the reading is completed.

\subsection*{Homework}

Practicing math by doing problems is the most effective way to learn the material. Homework will be posted on the course website and be due most Tuesdays in Thursdays by 9pm in the SQRC. Homework is graded for completeness and correctness. The following rubric is used when grading a homework problem:
\begin{itemize}\setlength\itemsep{0em}
    \item All aspects of the problem are addressed and a correct answer is given.
    \item All work is show and correct.
    \item Solutions are organized, legible, and easy to follow.
\end{itemize}
Each problem receives a score of 0,1 or 2. A score of 2 means the problem satisfied the criterion listed above, a score of 1 means that an effort was made to meet all criterion but one or more was not met, and a score of 0 indicates the problem is missing or there were severe shortcomings one or more of the criterion above.

Late homework will not be graded. Your lowest two homework scores for the semester will be dropped.

\subsection*{Quizzes}

Every week without an exam will have a quiz. The quizzes will cover the content of the previous week and are intended to give you continual feedback during the course. They also are good practice for the midterms and final.

Quizzes will typically be two or three questions covering the two or three sections covered in the previous week.

\subsection*{Exams}

There will be two midterms and a final.

\begin{itemize}
  \item Midterm 1: October 8th;
  \item Midterm 2: November 18th;
\end{itemize}

For all exams, you will be allowed to use both sides of hand written standard $8.5\times 11$ size paper. You may also use a non graphing calculator but you won't need it.

Midterms will attempt to have a question covering all the sections covered in class up until that point from the previous midterm. In a way they are like three quizzes stapled together. After each Midterm all students will complete a reflection on their performance in the course so far. More details will be given later.

The final will cover the new material from the three weeks after Midterm II but will be longer and also cover material from all previous sections.

To encourage continual improvement and mastery of the course goals, if a students midterm or final score on a question from section X is higher than their quiz score on a question from section X, the exam score will replace the quiz score.

\subsection*{Work Revisions}

All submitted work except, for pre-class questions, may be submitted for revision to receive up to half the missed points back. All revised work should be turned in to me at the beginning or end of class. Revised work must meet the following criterion to be regraded:
\begin{itemize}
    \item Revisions must be submitted within a week of receiving the work back.
    \item Revised work should be exemplary in both clarity and correctness.
    \item A short reflection must be included with all revised work. The reflection should address why the mistakes were made and what you can do to learn from them and improve in the future.
\end{itemize}

\subsection*{Collaboration}

Collaboration is an important part of learning mathematics and I strongly encourage you collaborate with your classmates on homework and studying for exams. With that said, there is a difference between working with someone else and copying down what they say or write without understanding it. I encourage you to write up final solutions on your own after you understand a problem, which might mean stepping away from your study group for 5 or 10 minutes.

\subsection*{Academic Integrity}

Any form of academic dishonesty or misconduct will not be tolerated. Cheating of any form will result in disciplinary actions, such as a zero grade or immediate failure of the course, and will be reported to the Office of the Dean.

\section*{Textbook}

The primary text for this course is \emph{Calculus: Early Transcendental Functions }, by Smith \& Minton, 4th Edition. We will cover most but not all of the sections in Chapters 1-4. A copy of the textbook is on reserve in Watzek Library. Course reserves are available on a first-come first-served basis for limited checkout periods of three hours.

There are many other free online resources for the calculus curriculum that cover roughly the same material in roughly the same order. In particular, AIM open textbook initiative at \url{https://aimath.org/textbooks/approved-textbooks/guichard/} has a list of free calculus textbooks and \url{https://www.khanacademy.org/math/ap-calculus-ab} has many videos and online practice problems.

If you are interested in using an alternative source besides the book let me know.

\section*{Technology}

I encourage you to use a computer or calculator on the homework and in class when appropriate. In fact, learning how to use such tools is an important skill for this class. However, it is also important to understand the computations and quizzes and exams will be written so no calculator is required. 

\section*{Academic Accommodations}

If you require academic accommodations please contact the Student Support Services Office in Albany Quadrangle (503-76-7192 or access@lclark.edu).  Once you complete the intake process and the Accommodations Agreement, you may Request to Send your Accommodations Letter.  Student Support Services staff will then notify faculty of the accommodations for which you are eligible.

\end{document}